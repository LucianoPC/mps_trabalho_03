\documentclass[a4paper, 11pt]{article}
\usepackage{comment} % enables the use of multi-line comments (\ifx \fi)
\usepackage{lipsum} %This package just generates Lorem Ipsum filler text.
\usepackage{fullpage} % changes the margin
\usepackage[brazilian]{babel}
\usepackage[utf8]{inputenc}
\usepackage[T1]{fontenc}
\usepackage{graphicx}
\usepackage{indentfirst}
\usepackage[table]{xcolor}

\begin{document}
\noindent
\large\textbf{Melhoria de Processo de Software / Terceiro trabalho}\\
Lucas Albuquerque Medeiros de Moura \hfill 11/0015568 \\
Luciano Prestes Cavalcanti \hfill 11/0035208

\section*{Relatório}

Este relatório visa identificar os problemas encontrados no processo do
projeto [nome do projeto], afim de estabelecer possíveis melhorias e assim
realizar uma contribuíção ao processo do projeto [nome do projeto].

As informações do relatório, que contém o processo avaliado com o que é
requerido e seu resultado esperado, juntamente com o problema e sugestões
de melhoria seguem na tabela \ref{tab:relatorio_de_avaliacao}.

\begin{table}[h]
\centering
\resizebox{\textwidth}{!}{\begin{tabular}{|l|l|l|l|l|}
\hline

\rowcolor[HTML]{EFEFEF}
\multicolumn{2}{|c|}{\textbf{Processo Avaliado}} &
\multicolumn{2}{|c|}{} \\ \hline

\rowcolor[HTML]{EFEFEF}
{\textbf{Requerido/Melhoria}} & {\textbf{Resultado Esperado}} &
{\textbf{Problema}} & {\textbf{Sugestão para Corrigir}} \\ \hline

\end{tabular}}
\caption{Relatório de Avaliação}
\label{tab:relatorio_de_avaliacao}
\end{table}

\end{document}
